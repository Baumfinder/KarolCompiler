\documentclass[ngerman]{article}

\usepackage{geometry}[top=1cm, bottom=1cm, left=1cm, right=1cm]
\usepackage{babel}[ngerman]
\usepackage{parskip}

\usepackage{csquotes}
\MakeOuterQuote{"}

\usepackage{titling}
\predate{}
\postdate{}

\usepackage{xcolor}
\definecolor{light-gray}{gray}{0.95}

\usepackage{listings}
\usepackage{lstautogobble}
\lstset{backgroundcolor=\color{light-gray}, basicstyle=\ttfamily, autogobble=true}

\title{KarolLang Dokumentation}
\author{Baumfinder}
\date{}

\begin{document}
    \maketitle

    Version: 1.1

    \section{Einleitung}
    Die KarolLang ist eine Sprache, die aus einer sehr dummen
    Idee entstanden ist, nämlich zu beweisen, dass RobotKarol
    turing-vollständig ist. Daraus entstand zunächt ein 
    Branfuck-Interpreter und danach eine (simple) Simulation
    einer theoretischen CPU in RobotKarol. Für die Instruktionen
    dieser CPU wurde erst ein Assembler und schließlich diese
    "high-level" Programmiersprache erstellt.

    \section{Die Programmiersprache}
    
    \subsection{Datentypen}
    Da in der CPU Daten in der Form von Zahlen gespeichert werden,
    gibt es nur Zahlen in dieser Programmiersprache. Diese Zahlen
    können auch negativ sein und Reichen von -999.999 bis 999999.
    Adressen sind nur positiv.
    
    Zusätzlich können Booleans durch 0="falsch" und 1="wahr"
    dargestellt werden. Alle anderen Zahlenwerte sind dann als
    Boolean ungültig, werden in den meisten (der einzigen)
    Implementation als "wahr" zugeordnet.

    \subsection{Variablen und Arrays}
    Variablen wird mit dem "var" Keyword erstellt. Arrays werden mit
    dem "arr" Keyword erstellt.

    Variablen kann mit einem "=" ein Wert zugewiesen werden, Arrays
    ebenfalls.

    Variablen und Arrays sind nur innerhalb ihres "Bereichs" aufrufbar,
    also kann man z.B. von Außerhalb nich auf die Variablen, die
    in einer Funktion definiert wurden, zugreifen.

    Beispiel:
    \begin{lstlisting}
        var a
        var b

        a = 5
        b = 69420

        arr b[10]
        b[3] = 666
    \end{lstlisting}

    Man kann mit dem Keyword "deref" anstatt einer Variablen dem
    Wert an einer Adresse einen Wert zuweisen.

    \subsection{Kommentare}
    Einzeilige Kommentare werden mit "//" begonnen.

    \subsection{Ausdrücke}
    In der Programmiersprache werden folgende Operatoren
    unterstützt:
    \begin{description}
        \item[a + b] addiert die beiden Zahlen
        \item[a - b] subtrahiert die beiden Zahlen
        \item[-a] negiert die Zahl
        \item[a * b] multipliziert die beiden Zahlen; Punkt vor
            Strich wird eingehalten
        \item[addr a] ist die Adresse der Variablen a
        \item[deref a] ist der Zahlenwert bei Adresse a
    \end{description}

    Außerdem gibt es Operatoren, die Zwei zahlen vergleichen. Die
    Ausgabe dieser Ausdrücke ist entweder 0 oder 1, da es keine
    Unterscheidung von Datentypen gibt. Deshalb kann man, im
    Gegensatz zu den meistem anderen Programmiersprachen, mit diesen
    dann als normale Zahlen weiterrechnen.
    \begin{description}
        \item[a == b] ist wahr, wenn a = b
        \item[a != b] ist wahr, wenn nicht a = b
        \item[a > b] ist wahr, wenn a > b
        \item[a < b] ist wahr, wenn a < b
    \end{description}

    Beispiel:
    \begin{lstlisting}
        var a
        var b
        a = (5 - 3) * 7
        // = 14
        b = a == 13 + 1
        // = 1, bzw. "wahr"
    \end{lstlisting}

    \subsection{Bedingte Anweisungen}
    Bedingte Anweisungen werden nur ausgeführt, wenn eine Bedingung
    wahr ist. Diese werden mit dem "if"-Keyword eingeleitet.

    \begin{lstlisting}
        // Syntax
        if *Bedingung* {
            *Anweisungen*
        }
    \end{lstlisting}
    \begin{lstlisting}
        // Beispiel
        if a == 4 {
            a = 14
        }
    \end{lstlisting}

    \subsection{Schleifen}
    Die while-Schleife wird mit dem Keyword "while" begonnen und
    wiederholt Instruktionen so lange, bis eine Bedingung nicht
    mehr wahr ist:
    \begin{lstlisting}
        // Syntax
        while *Bedingung* {
            *Instruktionen*
        }
    \end{lstlisting}
    \begin{lstlisting}
        // Beispiel
        var a
        a = 0
        while a < 10 {
            a = a + 1
        }
    \end{lstlisting}

    \subsection{Funktionen}
    Funktionen werden mit dem "func"-Keyword definiert, und mit ihrem
    Namen aufgerufen. Eine Funktion kann entweder ein oder keine
    Parameter entgegennehmen. Rekursion ist nicht erlaubt.
    Funtkionen können nur in ausdrücken und nicht als 
    eigenständige Anweisung aufgerufen werden.

    Innerhalb einer Funktion kann mit der "return"-Anweisung ein
    Wert zurückgegeben werden.

    \begin{lstlisting}
        // Syntax
        func *Name* ( *Parameter(optional)* ) {
            *Anweisungen*
        }
        
        // Aufruf
        var a
        a = *Name* ( *Parameter(optional)* )
    \end{lstlisting}
    \begin{lstlisting}
        // Beispiel
        func addiereFünf(a) {
            var b = a + 5
            return b
        }
        var ergebnis
        ergebnis = addiereFünf(45)
        // ergebnis ist 50
    \end{lstlisting}

    \section{Ein- und Ausgabe}
    In KarolLang gibt es keine explizite Ein- oder Ausgabe.
    Die Eingaben eines Programms wird im Quellcode festgelegt, und
    die Ausgaben können nach der Ausführung im Speicher der
    CPU angesehen werden.

\end{document}